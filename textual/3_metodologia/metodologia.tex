\chapter{Metodologia}

\section{Visão geral}

\textbf{Métodos:} A metodologia adotada para o desenvolvimento do robô equilibrista "Sir Galahad" segue as etapas abaixo:

\textbf{Definição de Requisitos:}
\begin{itemize}
   \item  Identificação dos requisitos funcionais e não-funcionais do robô, considerando busca, identificação e atingimento de alvos, além de equilíbrio dinâmico e precisão de movimento.
\end{itemize}

\textbf{Desenvolvimento do Software:}
\begin{itemize}
   \item Utilização da Raspberry Pi para implementação de partes do software de alto nível, utilizando a linguagem Python para o controle do robô e processamento de imagem.
   \item Integração do microcontrolador ESP32 para as camadas mais baixas do software, incluindo a leitura de dados dos sensores inerciais e controle dos motores.
\end{itemize}

\textbf{Modelagem Mecânica:}
\begin{itemize}
   \item Utilização do software SolidWorks para a modelagem 3D detalhada da estrutura do robô.
   \item Garantia de equilíbrio e resistência estrutural, além de validação da ergonomia e das dimensões do robô virtualmente antes da fabricação.
\end{itemize}

\textbf{Fabricação e Montagem:}
\begin{itemize}
   \item Fabricação da estrutura mecânica por meio de manufatura aditiva, assegurando leveza e precisão.
   \item Montagem dos componentes eletrônicos na estrutura física, levando em conta distribuição de peso e acessibilidade para manutenção.
\end{itemize}

\textbf{Projeto Eletrônico:}
\begin{itemize}
   \item Utilização do software Flux para projetar o circuito eletrônico, incluindo os componentes necessários para sensores, microcontroladores e motores.
   \item Garantia de conexões adequadas e otimização da eficiência energética.
\end{itemize}

\textbf{Testes e Ajustes:}
\begin{itemize}
   \item Realização de testes contínuos para verificar a capacidade de equilíbrio, precisão de movimento e detecção de alvos.
   \item Ajustes nos algoritmos de controle e nos parâmetros dos sensores para aprimorar o desempenho do robô.
\end{itemize}

\textbf{Validação e Avaliação:}
\begin{itemize}
   \item Validação final por meio de testes de campo e avaliação dos resultados obtidos em relação aos objetivos propostos.
   \item Comparação do desempenho do "Sir Galahad" com os critérios estabelecidos e considerações finais sobre o projeto.

\end{itemize} 

\section{Projeto mecânico}

TODO: inserir projeto mecânico

\section{Projeto de hardware}

TODO: inserir projeto de hardware e eletronica

* Obs.: nao esqueça de apresentar o diagrama de blocos do hardware. 


\section{Projeto de software}

 TODO: inserir aqui os diagramas de software

* Obs.: nao esqueça de apresentar os diagramas de estados (statecharts) do 
software.

\section{Integração}

TODO: discorrer sobre a integração