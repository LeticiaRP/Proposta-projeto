% definição tipo de documento
\documentclass[
12pt,       % tamanho da fonte
openright,  % inicio documento a direita
oneside,    % frente e verso
a4paper,    % folha a4
sumario = abnt-6027-2012, % Formatação do sumário: tradicional (estilo tradicional) ou abnt-6027-2012 (norma ABNT 6027-2012)
chapter = TITLE,          % Títulos de capítulos em maiúsculas 
%section = TITLE,
pretextualoneside,        % Impressão dos elementos pré-textuais pretextualoneside em um lado da folha
fontetimes,               % Fonte do texto: (times)
% fontearial,                % Fonte do texto: (arial) 
% fontecourier            % Fonte do texto: (courier), 
% fontemodern,              % Fonte do texto: (lmodern - default latex)
semrecuonosumario,        % Remoção do recuo dos itens no sumário (comente para adição do recuo, se estilo tradicional)
legendascentralizadas,    % Alinhamento das legendas centralizado (comente para alinhamento à esquerda)
pardeassinaturas,         % Assinaturas na folha de aprovação em até duas colunas (comente para em uma única coluna)
linhasdeassinaturas,      % Linhas de assinaturas na folha de aprovação (comente para remover as linhas)
english,french,spanish,
brazil      % o ultimo idioma é o padrão do documento
]{utfprct}
% -----------------------------------------------------
% pacotes utilizados no documento
\usepackage[utf8]{inputenc} % codificação texto
\usepackage[T1]{fontenc}    % codificação para adicionar acentuação
\usepackage[brazil]{babel}  % suporte para portugues
\usepackage{fancyhdr}       % cabeçalhos e notas de rodapé
\usepackage{graphicx}       % incluir figuras
\usepackage{geometry}       % configurar margens do documento
\usepackage{indentfirst}    % Indenta o primeiro parágrafo de cada seção.
% \usepackage{lmodern}        % fonte Latin Modern 
\usepackage{bigdelim, booktabs, colortbl, longtable, multirow}      % Ferramentas para tabelas
\usepackage{amssymb, amstext, amsthm, icomma}                       % Ferramentas para linguagem matemática
\usepackage{pifont, textcomp, wasysym}                              % Simbolos texto
\usepackage[normalem]{ulem}                                         % Sublinhar texto
\usepackage[alf]{abntex2cite}                                       % Citações padrão ABNT


% -----------------------------------------------------
% comandos adicionais
\newcommand{\cpp}{\texttt{C$++$}}       % escrever C++
\newcommand{\ds}{\displaystyle}         % Tamanho normal das equações
\newcommand{\bsym}[1]{\boldsymbol{#1}}  % Texto no modo matemático em negrito
\newcommand{\mr}[1]{\mathrm{#1}}        % Texto no modo matemático normal (não itálico)
\newcommand{\pare}[1]{\left(#1\right)}  % Parênteses
\newcommand{\colc}[1]{\left[#1\right]}  % Colchetes
\newcommand{\nomeequacao}{Equação}
\newcommand{\nomeequacoes}{Equações}


% -----------------------------------------------------
% espaçamento entre um paragrafo e outro
\setlength{\parskip}{0.2cm}

% % -----------------------------------------------------
% % ajuste de margens no documento 
% \setlrmarginsandblock{3cm}{2cm}{*}
% \setulmarginsandblock{3cm}{2cm}{*}
% \checkandfixthelayout

% -----------------------------------------------------
% informações do tipo de documento
\TipoDeDocumento{Trabalho de Conclusão de Curso}
\NivelDeFormacao{Bacharelado}
\TituloPretendido{Bacharel}

% -----------------------------------------------------
% informações da monografia

% ------- título
% Título do documento em PORTUGUÊS (resumo)
\TituloEmMultiplasLinhas{Proposta de Projeto - \textit{Sir Galahad}}%d

% Título do trabalho em INGLÊS (abstract)
\TituloEmMultiplasLinhasIngles{Project proposal - Sir Galahad}

% ------- autora da monografia
\NomeDoAutor{Guilherme Pires Silva}
\SobrenomeDoAutor{Silva}
\PrenomeDoAutor{Guilherme Pires}

\AtribuiAutorDois{true}%%
\NomeDoAutorDois{Letícia Rodrigues Pinto}
\SobrenomeDoAutorDois{Pinto}
\PrenomeDoAutorDois{Letícia Rodrigues}

\AtribuiAutorTres{false}%%

% ------- orientador
\AtribuicaoOrientador{Orientador}%% Atribuição "Orientador(a)"
\TituloDoOrientador{Prof.}%% Título do(a) orientador(a)
\NomeDoOrientador{Daniel Rossato de Oliveira}%% Nome completo do(a) orientador(a)

% ------- coorientador
%% Usado para citação: "\SobrenomeDoCoorientador, PrenomeDoCoorientador" (ex: "Doe, John" ou "Doe, J.")
\AtribuiCoorientador{false}%% Insere ou remove o(a) coorientador(a): "true" ou "false"

% ------ instituição
\Instituicao{Universidade Tecnológica Federal do Paraná}
\Institution{Universidade Tecnológica Federal do Paraná} % [abstract] Institution name (*nome sem traduzir é o recomendado para docs. da UTFPR)
\SiglaInstituicao{UTFPR}

\Departamento{Departamento Acadêmico de Eletrônica}
\SiglaDepartamento{DAELN}
\Curso{Engenharia da Computação, Engenharia Elétrica e Engenharia Mecatrônica}
\Course{Computer Engineering, Electrical Engineering and Mechatronics Engineering}

\Cidade{Curitiba}

\Ano{2023}

% ---------------------------------------------------------------------
% folha de rosto
\DescricaoDoDocumento{
Proposta de projeto apresentado como requisito para aprovação na disciplina Oficina De Integração 2, do \imprimirppgoudepartamento, da \imprimirinstituicao\ (\imprimirsiglainstituicao).
}

% -----------------------------------------------------

\begin{document}

% pre textual ------------------------------------------
% inclui capa do documento
\incluircapa

% formatação da pagina dos elementos pre-textuais
\pretextual

% tipo de licença na que aparece na folha de rosto
\Licenca{CC-BY}

% incluir folha de rosto
\incluirfolhaderosto

% resumo
% resumo

\begin{resumo}% Ambiente resumo

    
    Neste trabalho, apresentamos o projeto \textit{Sir Galahad}, um robô equilibrista de duas rodas que incorpora conceitos avançados de robótica e controle de sistemas dinâmicos. O objetivo central deste projeto é desenvolver um robô capaz de manter seu equilíbrio enquanto se movimenta em diferentes ambientes e atinge alvos predefinidos com precisão.

A busca por soluções autônomas e versáteis na robótica tem impulsionado o desenvolvimento de sistemas que podem interagir de maneira sofisticada com o mundo ao seu redor. O \textit{Sir Galahad} é um exemplo de como a integração de tecnologias complexas pode resultar em um sistema que se assemelha ao equilíbrio de um pêndulo invertido

Ao longo deste trabalho apresentaremos os requisitos funcionais e não funcionais do projeto, além de indicar como esperamos desenvolver este robô no decorrer deste semestre, exibindo materiais e métodos esperados, além de possíveis riscos e seus respectivos planos de ação. 


    %\vspace*{\offinterlineskip} % pular uma linha
    \noindent
    \textbf{Palavras-chaves}:robótica; sistemas de controle; robô equilibrista; integração de sistemas; pêndulo invertido; sensores.

\end{resumo}
    

% abstract
% Abstract

\begin{resumo}[Abstract]% Ambiente abstract

    \begin{otherlanguage}{english}
        
        In this work, we present the \textit{Sir Galahad} project, a two-wheeled balancing robot that incorporates advanced concepts in robotics and dynamic systems control. The central objective of this project is to develop a robot capable of maintaining its balance while navigating diverse environments and accurately targeting predefined objectives.

        The pursuit of autonomous and versatile solutions in robotics has driven the development of systems that can interact intricately with their surroundings. The \textit{Sir Galahad} project serves as an illustration of how the integration of complex technologies can yield a system akin to the balance of an inverted pendulum.
        
        Throughout this work, we will introduce the project's functional and non-functional requirements, along with our approach to developing this robot over the course of the semester. We will outline expected materials and methods, as well as potential risks and corresponding action plans.
    
        %\vspace*{\offinterlineskip} % pular uma linha
        \noindent
        \textbf{Keywords}: robotics; control systems; balancing robot; systems integration; inverted pendulum; sensors.

    
    \end{otherlanguage}

\end{resumo}
    

% lista de ilustrações
\incluirlistadeilustracoes

% incluir sumário
\incluirsumario 

% textual ------------------------------------------------
\textual

% capítulo 1 - introdução
% capítulo 1 - introdução
% ------------------------------
% estrutura do capítulo
% 1. Introdução
% 1.1 Considerações iniciais
% 1.2 Objetivos
%     1.2.1 Objetivo geral 
% 1.3 Justificativa
% 1.4 Declaração do escopo de alto nível
%     1.4.1 Requisitos funcionais
%     1.4.2 Requisitos não-funcionais
% 1.5 Materiais e Métodos
% 1.6 Integração
% 1.7 Análise de Riscos
% 1.8 Estrutura do trabalho
% 1.9 Cronograma
% -------------------------------

% 1 ------------- introdução
\chapter{Introdução}\label{cap:introducao}

Nesta seção introdutória apresentamos o projeto \textit{Sir Galahad}, um robô equilibrista de duas rodas, uma ideia que une a robótica e o controle de sistemas dinâmicos. Nosso enfoque reside em desenvolver um robô capaz de manter seu equilíbrio enquanto se movimenta em ambientes diversos. Ao longo deste documento, examinaremos os objetivos, justificativas e escopo do projeto, além de destacar a importância desse tipo de tecnologia na busca por soluções autônomas e versáteis.

% 1.1 ----------- considerações iniciais
\section{\textbf{Considerações iniciais}}

Nesta seção, exploramos o contexto que motiva a criação do projeto \textit{Sir Galahad}. A busca por um robô equilibrista de duas rodas, capaz de buscar e atingir um alvo, de forma que este trabalho combine conceitos, como de controle dinâmico, fusão de sensores e integração de sistemas. 

A criação de um robô que equilibra-se sobre duas rodas, de forma análoga a um pêndulo invertido, desafia os paradigmas convencionais da mobilidade robótica. Esta iniciativa alinha-se à busca por soluções tecnológicas que, além de atender necessidades práticas, inspiram a exploração dos limites da engenhosidade humana. Neste contexto, a motivação diante da escolha do projeto  \textit{Sir Galahad} é permeada por questões técnicas, científicas e criativas. 

Este trabalho se propõe a detalhar o projeto, desde suas premissas até sua concretização. Ao explorar as considerações iniciais que permeiam \textit{Sir Galahad}. 

% 1.2 ----------- objetivos
\section{\textbf{Objetivos}}
Os objetivos deste projeto refletem uma busca coerente por resultados concretos que aliam a funcionalidade técnica com as expectativas criativas. Compreender e atender esses objetivos é fundamental para avaliar o sucesso do projeto proposto.

% 1.2.1 ----------- objetivo geral
\subsection{Objetivo geral}
% no maximo 3 linhas 
O objetivo geral deste trabalho é o projeto e implementação de um robô diferencial de duas rodas, equilibrista, que deve ser capaz de procurar e identificar um alvo (previamente definido),  combinando controle dinâmico, sensoriamento e integração de sistemas.

% 1.3 ----------- justificativa
\section{\textbf{Justificativa}}

A criação do projeto \textit{Sir Galahad} é motivada pela busca por avanços na robótica e suas aplicações práticas. Ao  desenvolver um equilibrista de duas rodas, podemos explorar o potencial da robótica em situações desafiantes, contribuindo para o conhecimento de controle e sensoriamento.

A capacidade do \textit{Sir Galahad} de buscar, identificar e atingir alvos tem implicações práticas em inspeção e monitoramento, além de ser uma expressão da criatividade humana na tecnologia robótica.

\section{\textbf{Integração}}
Discutiremos aqui as matérias cujos conceitos são de suma importância para o desenvolvimento do projeto. 
\begin{itemize}
   \item Para o desenvolvimento do \textit{software}
      \begin{itemize}
         \item Fundamentos de programação
         \item Análise e projeto de sistemas
         \item Engenharia de \textit{software}
         \item Robótica móvel 
         \item Sistemas inteligentes
      \end{itemize}
   \item Para o desenvolvimento da eletrônica
      \begin{itemize}
         \item Eletrônica geral
         \item Circuitos digitais
         \item Desenho técnico aplicado 
         \item Desenho eletrônico
      \end{itemize}
   \item Para a sintonização do sistema de controle: 
      \begin{itemize}
         \item Controle 1 
         \item Tópicos especiais em controle 
      \end{itemize}
   \item Para o desenvolvimento da mecânica
      \begin{itemize}
         \item Desenho técnico 
         \item Desenho de máquinas 1
         \item Desenho de máquinas 2
      \end{itemize}
\end{itemize}


% 1.4 ----------- declaração do escopo de alto nível
\section{\textbf{Declaração do escopo de alto nível}}

O protótipo criado no projeto consiste em um veículo autônomo que deve detectar um alvo ou objetivo por meio de reconhecimento de imagem e se locomover em direção à ele enquanto se equilibra sobre duas rodas como um pêndulo invertido.
Não está dentro do escopo do projeto a navegação em ambientes externos, sobre terrenos não planos ou que possuam quaisquer tipos de obstáculos que possam obstruir a navegação.

% 1.4.1 ----------- requisitos funcionais
\subsection{Requisitos funcionais}

RF1: Se manter em equilíbrio sobre duas rodas estando parado ou em movimento em uma superfície plana em um ambiente fechado.

RF2: Detectar um alvo de formato, tamanho e cor previamente determinados em um ambiente bem iluminado.

RF3: Se locomover em direção ao alvo designado contanto que o terreno seja regular e não existam obstáculos em seu caminho.

RF4: Sinalizar que chegou perto ou atingiu o alvo/objetivo.


% 1.4.2 ----------- requisitos não-funcionais
\subsection{Requisitos não-funcionais}

RNF1: O controle do equilíbrio será feito utilizando dados captados por sensores como acelerômetro, giroscópio e magnetômetro.

RNF2: O alvo deve ser detectado utilizando processamento de imagem.

RNF3: Inicialmente o alvo a ser detectado será uma bola de tênis.

RNF4: O controle e processamento de imagem será feito por um Raspberry Pi 4.

RNF5: Os sensores serão controlados por um ESP32.

RNF6: Ao se aproximar ou atingir o alvo, o robô deverá fazer um alerta visual para informar que a operação foi concluída.

RNF7: A linguagem utilizada para controlar o robô será Python.

RNF8: Os dois motores serão acionados por Ponte H que será controlada pelo ESP32.

% 1.5 ----------- materiais e métodos
\section{\textbf{Materiais e métodos}}
Nesta seção, descrevemos os materiais e custos esperados, além da abordagem metodológica adotada para o desenvolvimento do robô equilibrista.

\textbf{Materiais:} Abaixo estão listados os materiais previstos para a realização do projeto, os itens destacados com (*) já são de posse dos membros do projeto, portanto não somaram aos custos esperados. 

\begin{tabframed}[h]%% Ambiente tabframed
   %\captionsetup{width=0.5\textwidth}%% Largura da legenda
   \caption{Materiais utilizados no desenvolvimento do sistema}%% Legenda
   \label{quad:exemplo1}%% Rótulo
   \renewcommand{\arraystretch}{1.5}
   \begin{tabular}{|l|l|l|l|l}
   \cline{1-4}
   \textbf{Materiais} & \textbf{Valor unitário (R\$)} & \textbf{Quantidade} & \textbf{Valor total (R\$)} \\ 
    
      \hline
      \href{https://produto.mercadolivre.com.br/MLB-2004934384-lip0-3s-111-v-1500-mah-3s-_JM?matt_tool=68334988&matt_word=&matt_source=google&matt_campaign_id=14300471977&matt_ad_group_id=127503848075&matt_match_type=&matt_network=g&matt_device=c&matt_creative=542969655996&matt_keyword=&matt_ad_position=&matt_ad_type=pla&matt_merchant_id=542516090&matt_product_id=MLB2004934384&matt_product_partition_id=1801247246545&matt_target_id=pla-1801247246545}{Power Banck 1000mAh} & 84,90 & 1 & 84,90 \\

      \hline 
      \href{https://www.aliexpress.com/item/32223093678.html?srcSns=sns_Copy&spreadType=socialShare&bizType=ProductDetail&social_params=21099436311&aff_fcid=c234ac5e17824f20a319f9cc17dbaf2f-1692194363304-03902-_mPjn7Ro&tt=MG&aff_fsk=_mPjn7Ro&aff_platform=default&sk=_mPjn7Ro&aff_trace_key=c234ac5e17824f20a319f9cc17dbaf2f-1692194363304-03902-_mPjn7Ro&shareId=21099436311&businessType=ProductDetail&platform=AE&terminal_id=676ed690bcd2403d8bbab55c9f2e36b3&afSmartRedirect=y}{Drive Motor DRV8833} & 2,61 & 2 &	5,22 \\
   
      \hline 
      \href{https://shopee.com.br/product/534679327/14372089266}{(*) Micro Servo MG90S}	& 13,00 & 1 & 13,00 \\

      \hline
      \href{https://www.makerhero.com/produto/raspberry-pi-4-model-b/}{(*) Raspberry Pi 4 Model B+} & 698,15 & 1 & 698,15 \\

      \hline
      \href{https://shopee.com.br/product/550918841/11054519654}{(*) Esp32} & 23,31 & 1 & 23,31 \\

      \hline
      Custos de impressão 3D & 75,00 & 1 & 75,00 \\

      \hline
      \href{https://pt.aliexpress.com/item/1005001279982165.html?srcSns=sns_Copy&spreadType=socialShare&bizType=ProductDetail&social_params=21106393547&aff_fcid=0a6046e63fd945d7a1d698fe1e129e32-1692194250357-01162-_mMVW6Jk&tt=MG&aff_fsk=_mMVW6Jk&aff_platform=default&sk=_mMVW6Jk&aff_trace_key=0a6046e63fd945d7a1d698fe1e129e32-1692194250357-01162-_mMVW6Jk&shareId=21106393547&businessType=ProductDetail&platform=AE&terminal_id=676ed690bcd2403d8bbab55c9f2e36b3&afSmartRedirect=y}{Motor com encoder e roda JGA25-370}	& 34,25	 & 2 & 68,50 \\

      \hline
      \href{https://produto.mercadolivre.com.br/MLB-2004934384-lip0-3s-111-v-1500-mah-3s-_JM?matt_tool=68334988&matt_word=&matt_source=google&matt_campaign_id=14300471977&matt_ad_group_id=127503848075&matt_match_type=&matt_network=g&matt_device=c&matt_creative=542969655996&matt_keyword=&matt_ad_position=&matt_ad_type=pla&matt_merchant_id=542516090&matt_product_id=MLB2004934384&matt_product_partition_id=1801247246545&matt_target_id=pla-1801247246545}{(*) Bateria LiPo 3s 1500 mAh} & 119,99 & 1 & 119,99 \\

      \hline
      \href{https://pt.aliexpress.com/item/1005005915264178.html?srcSns=sns_Copy&spreadType=socialShare&bizType=ProductDetail&social_params=21100114064&aff_fcid=9b62215bdd524891bba0b98b5bbb55e5-1692194179415-09393-_mKQfDDM&tt=MG&aff_fsk=_mKQfDDM&aff_platform=default&sk=_mKQfDDM&aff_trace_key=9b62215bdd524891bba0b98b5bbb55e5-1692194179415-09393-_mKQfDDM&shareId=21100114064&businessType=ProductDetail&platform=AE&terminal_id=676ed690bcd2403d8bbab55c9f2e36b3&afSmartRedirect=y}{Sensor Inercial BNO055 + BMP280, SEN0253} & 361,40 & 1 & 361,40 \\
    
      \hline
      \href{https://pt.aliexpress.com/item/1005003954117993.html?srcSns=sns_Copy&spreadType=socialShare&bizType=ProductDetail&social_params=21106405512&aff_fcid=ecdaa74a5af14122ba08116816bff22f-1692194368880-06711-_mttWAZ0&tt=MG&aff_fsk=_mttWAZ0&aff_platform=default&sk=_mttWAZ0&aff_trace_key=ecdaa74a5af14122ba08116816bff22f-1692194368880-06711-_mttWAZ0&shareId=21106405512&businessType=ProductDetail&platform=AE&terminal_id=676ed690bcd2403d8bbab55c9f2e36b3&afSmartRedirect=y}{Câmera}	& 59,32	& 1 & 59,32 \\

      \hline

   \end{tabular}
   \fonte{Autoria própria}%% Fonte
   \end{tabframed}

\textbf{Métodos:} A metodologia adotada para o desenvolvimento do robô equilibrista "Sir Galahad" segue as etapas abaixo:

\textbf{Definição de Requisitos:}
\begin{itemize}
   \item  Identificação dos requisitos funcionais e não-funcionais do robô, considerando busca, identificação e atingimento de alvos, além de equilíbrio dinâmico e precisão de movimento.
\end{itemize}

\textbf{Desenvolvimento do Software:}
\begin{itemize}
   \item Utilização da Raspberry Pi para implementação de partes do software de alto nível, utilizando a linguagem Python para o controle do robô e processamento de imagem.
   \item Integração do microcontrolador ESP32 para as camadas mais baixas do software, incluindo a leitura de dados dos sensores inerciais e controle dos motores.
\end{itemize}

\textbf{Modelagem Mecânica:}
\begin{itemize}
   \item Utilização do software SolidWorks para a modelagem 3D detalhada da estrutura do robô.
   \item Garantia de equilíbrio e resistência estrutural, além de validação da ergonomia e das dimensões do robô virtualmente antes da fabricação.
\end{itemize}

\textbf{Fabricação e Montagem:}
\begin{itemize}
   \item Fabricação da estrutura mecânica por meio de manufatura aditiva, assegurando leveza e precisão.
   \item Montagem dos componentes eletrônicos na estrutura física, levando em conta distribuição de peso e acessibilidade para manutenção.
\end{itemize}

\textbf{Projeto Eletrônico:}
\begin{itemize}
   \item Utilização do software Flux para projetar o circuito eletrônico, incluindo os componentes necessários para sensores, microcontroladores e motores.
   \item Garantia de conexões adequadas e otimização da eficiência energética.
\end{itemize}

\textbf{Testes e Ajustes:}
\begin{itemize}
   \item Realização de testes contínuos para verificar a capacidade de equilíbrio, precisão de movimento e detecção de alvos.
   \item Ajustes nos algoritmos de controle e nos parâmetros dos sensores para aprimorar o desempenho do robô.
\end{itemize}

\textbf{Validação e Avaliação:}
\begin{itemize}
   \item Validação final por meio de testes de campo e avaliação dos resultados obtidos em relação aos objetivos propostos.
   \item Comparação do desempenho do "Sir Galahad" com os critérios estabelecidos e considerações finais sobre o projeto.

\end{itemize}

% 1.7 ----------- analise de riscos 
\section{\textbf{Análise de riscos}}
A análise de riscos desempenha um papel crucial na identificação e previsão de possíveis problemas que podem surgir durante o desenvolvimento do projeto. Na Tabela de Riscos, realizamos um levantamento abrangente dos principais problemas, avaliando cada um deles com base em critérios essenciais. Esses critérios incluem a probabilidade de ocorrência, a gravidade e o impacto do problema, bem como a facilidade de resolução e sua viabilidade.

Aqui está uma explanação detalhada de cada coluna da Tabela de Riscos:

\begin{itemize}
   \item \textbf{Identificadores e Descrições:} As primeiras duas colunas numeram e descrevem cada possível falha ou problema que pode afetar o sistema.
   \item \textbf{Probabilidade de Ocorrência:} A terceira coluna avalia a probabilidade de o problema acontecer, variando de "1" (baixa probabilidade) a "5" (alta probabilidade).
   \item \textbf{Gravidade do Problema:} A quarta coluna atribui um valor de gravidade ao problema, indo de "1" (impacto mínimo) a "5" (impacto significativo).
   \item \textbf{Dificuldade de Resolução:} A quinta coluna reflete a facilidade ou dificuldade em resolver o problema, variando de "1" (dificuldade alta) a "5" (dificuldade baixa).
   \item \textbf{Estratégia de Resolução:} A coluna subsequente sugere possíveis medidas para lidar com cada problema após sua ocorrência. Essas medidas visam resolver o problema, mas o projeto também adota ações preventivas para antecipar sua aparição.
   \item \textbf{Viabilidade Individual:} A última coluna representa a viabilidade de cada problema, calculada como o produto da probabilidade e da gravidade. Se esse valor for 13 ou superior, a viabilidade do projeto pode ser comprometida. Nesse caso, a solução proposta deve ser reavaliada para garantir uma relação mais favorável entre probabilidade e impacto, aumentando a viabilidade e reduzindo o risco.
   
\end{itemize}

No Quadro \ref*{quad:analise_riscos} se encontra o levantamento de riscos que podem ou não ocorrer ao longo do desenvolvimento do projeto, e o plano de ação respectivo em caso. 

\begin{tabframed}[h]
   \caption{Análise de Riscos}
   \label{quad:analise_riscos}
   \renewcommand{\arraystretch}{1.5}
   \small
   \begin{tabular}{|l|p{2.5cm}|l|l|l|p{2.5cm}|l|}
   \cline{1-7}
   \textbf{N°} & \textbf{Risco} & \textbf{Probabilidade} & \textbf{Gravidade} & \textbf{Resolução}  &  \textbf{Estratégia de ação} &  \textbf{Viabilidade} \\ \cline{1-7}
    1 & Falha na Detecção de Alvos & 4 & 3 & 3 & Aprimorar algoritmos de detecção, considerar redundância de sensores & Viável \\ \cline{1-7}
    2 & Dificuldades de Equilíbrio & 3 & 4 & 4 & Desenvolver algoritmos robustos de controle de equilíbrio, otimizar distribuição de peso & Viável \\ \cline{1-7}
    3 & Colisão com Obstáculos & 3 & 3 & 3 & Implementar sensores de proximidade e algoritmos de evasão de obstáculos & Viável \\ \cline{1-7}
    4 & Complexidade de Integração de Hardware e Software & 4 & 3 & 4 & Abordagem modular, testes frequentes de integração & Viável \\ \cline{1-7}
    5 & Desafios na Fabricação de Componentes & 2 & 4 & 3 & Colaboração com especialistas, prototipagem e iterações & Viável \\ \cline{1-7}
    6 & Incompatibilidade entre Componentes Eletrônicos & 3 & 3 & 4 & Testes prévios de compatibilidade, uso de componentes confiáveis & Viável \\ \cline{1-7}
    7 & Queima de Componentes Eletrônicos & 3 & 4 & 3 & Proteção contra picos de tensão, testes de carga elétrica & Viável \\ \cline{1-7}
    8 & Atraso na Entrega de Componentes & 4 & 3 & 3 & Pedidos antecipados, comunicação com fornecedores, estoque de segurança & Viável \\ \cline{1-7}
    9 & Calibração Incorreta dos Sensores & 3 & 3 & 3 & Implementar procedimentos de calibração rigorosos, verificar e ajustar regularmente os sensores & Viável \\ \cline{1-7}
    10 & Interferência de Ambiente & 2 & 2 & 3 & Realizar testes em diferentes ambientes, isolar componentes sensíveis & Viável \\ \cline{1-7}
   \end{tabular}
   \fonte{Autoria própria}
   \end{tabframed}


% 1.9 ----------- cronograma 
\section{\textbf{Cronograma}}
As informações sobre o projeto estão centralizadas em sua página do \href{https://www.notion.so/Sir-Galahad-d9482e0c1ac040d4a7b6bf2bfb223bff}{\textcolor{blue}{Notion}}, e o cronograma pode ser visto em detalhes por meio do \href{https://www.notion.so/59f60f78a3d144e7a52582be31b53f8d?v=2f55205b8da0482ebcb1618c6ef6c0b5}{\textcolor{blue}{link}}. 


% capítulo 2 - fundamentação teórica

%! ---------- CAPÍTULO 2 -> FUNDAMENTAÇÃO TEÓRICA
\chapter{Fundamentação teórica}

O pêndulo invertido é um sistema dinâmico clássico amplamente estudado em teoria de controle e robótica. Ele consiste em um braço vertical, representando o pêndulo, montado em um ponto móvel horizontal \cite{Kajita2001}. Este sistema é intrinsecamente instável, pois pequenos desvios de sua posição vertical resultam em forças que ampliam esses desvios. \cite{Ogata2009}

O desafio fundamental associado ao pêndulo invertido reside na capacidade de manter o equilíbrio em uma posição vertical, apesar da instabilidade inerente. Para controlar eficazmente um pêndulo invertido, são necessários algoritmos de controle que possam ajustar continuamente a posição do ponto de montagem para compensar quaisquer perturbações \cite{Spon2005}.

O \textit{Sir Galahad}, sendo um robô equilibrista de duas rodas, compartilha princípios fundamentais com o pêndulo invertido. Ao mover-se, ele deve constantemente ajustar sua posição para evitar quedas. Isso requer algoritmos de controle sofisticados que levam em consideração variáveis como aceleração, inclinação e feedback sensorial para manter o equilíbrio dinâmico.


%! --------- 2.2 -> 









\section{Controle}

Para que o robô seja capaz de se equilibrar é preciso que exista algum tipo de controle, neste caso o controlador integral derivativo (PID). A ideia básica para o PID é regular propriedades de um sistema através da informação fornecida por um sensor. Para realizar isso o PID utiliza da combinação de três fatores: Proporcional, Integral e Derivativo.
Na resposta do proporcional, tem-se o fator erro sendo multiplicado por uma constante Kp. Esta constante determina a taxa de resposta de saída para o sinal. Na resposta integral tem-se uma constante Ki como fator de proporcionalidade ao erro acumulado num determinado período de tempo, por isso integrativo. Enquanto na resposta derivativa tem-se então uma constante Kd que representa a magnitude da variação do erro ao controle.
O erro consiste na diferença entre o valor de referência, também conhecido como \textit{set point}, e o valor dado pelo sensor ou variável de processo. Outros termos que são importantes quando analisarmos o desempenho e comportamento do controle são:
\textit{Rise Time} – Tempo que demora para o sistema leva para ir de 10\% a 90\% do \textit{set point}
\textit{Overshoot} – Valor em que a variável de processo ultrapassa o \textit{set point}, também pode ser expresso em porcentagem
\textit{Setlling Time} – Tempo que demora para a variável de processo estabilizar num valor próximo (normalmente em até 5\%) do \textit{set point}
\textit{Steady-State Error} – Diferença final entre o \textit{set point} e a variável de processo.
considerando todos esses fatores tem-se uma prévia sobre a influência dos valores destes parâmetros para o funcionamento da função de equilíbrio do motor.

\subsection{PWM}

Uma parte muito relevante do projeto se da exatamente em como controlar a velocidade dos motores para que o robô se comporte de uma maneira especifica, neste caso os dois motores DC devem manter o equilíbrio e o servo motor deve subir e descer de acordo com a presença do alvo.

Para esse controle é usado então a modulação por largura de pulso (PWM). O PWM, de forma simples, consiste em determinar por quanto tempo um certo dispositivo ou componente fica "ligado" controlando a potencia entregue ao mesmo. O PWM é dado pela equação abaixo:

\begin{equation}
    DutyCycle = \frac{100\times L_p}{T}
\label{PWM}
\end{equation}

Onde:

\begin{itemize}
\item DutyCycle - Valor expresso em porcentagem e demonstra o valor médio da potencia em relação ao seu valor de máximo.
\item $L_p$ - Largura do pulso, tempo em que o sinal está "ligado".
\item T - Período, duração para um ciclo de onda.
\end{itemize}

Para os motores DC a potencia aplicada a eles é diretamente proporcional a velocidade e através do DutyCycle é possível manipular a potencia média do motor, ou seja, para controlar a velocidade basta deixar o motor "ligado" por um determinado tempo que seja de interesse.

Já no caso do servo motor a LarguraDoPulso determina a posição (angulo) em que ele se encontra. Os valores para as posições, na maioria dos servo motores, são padronizadas para 0,6 ms para $0º$, 1,5 ms para $90º$ e 2,4 para $180º$, tudo isso num Periodo de 20 ms.

\section{Sensores}

Uma importante parte sobre os dispositivos eletrônicos é o uso de sensores. Eles são responsáveis por obter informações do meio, com as quais um equipamento pode ser realmente considerado um dispositivo capaz de interagir de forma inteligente e responsiva. Para este projeto foram usados uma câmera, giroscópio e acelerômetro.

\subsection{Câmera}

A obtenção de imagens foi um fator essencial para o funcionamento do protótipo. Responsável pela identificação do alvo, feita através do processamento digital das imagens capturadas.

Para que o protótipo seja capaz de reconhecer seu alvo foi preciso que seja programado para tal tarefa. Neste caso será utilizada Python e a biblioteca OpenCV, que nos possibilita trabalhar com imagens e vídeos.

Como resultado da pesquisa para reconhecer imagens ficou evidente alguns conceitos importantes afim de concretizar a tarefa. São estes:

\begin{itemize}
    \item Binarização de imagem
    \item Suavização de imagem
\end{itemize}

A binarização de imagem é um conceito simples que se trata de criar uma segunda imagem, uma máscara, onde se faz uma análise na primeira imagem e todos os pixels que satisfazerem certa condição se tornam brancos (1), enquanto os demais ficam pretos (0). Assim, o resultado é uma máscara em preto e branco que facilita a manipulação ou tratamento da imagem original. Neste caso a condição para a binarização consiste na cor do pixel estar dentro de uma faixa de cor desejada ou não.

A suavização de imagem se trata da aplicação de algum filtro sobre a imagem, como por exemplo borrar. Existem diversos tipos de filtros que podem ser aplicados sobre as imagens, mas aqui estamos interessados em tratar a máscara obtida na binarização afim de melhorar o resultado final, então temos algumas operações como Erosion, Dilation, Opening, Closing e Morphological Gradient, todas disponíveis na OpenCV.

%TODO: IMAGEM TESTE DE RECONHECIMENTO
% \begin{figure}[!htb]
% 	\centering
% 	\includegraphics[width=0.6\textwidth]{camera.png}% <- formatos PNG, JPG e PDF
% 	\caption[Imagem de teste]{Imagem gerada no teste de reconhecimento.}
% 	\fonte{Autoria própria.}
% 	\label{fig:sbroriginal}
% \end{figure}

\subsection{Sensor inercial}

Para obtenção dos dados necessários à regulagem dos motores das rodas, necessitou-se o uso de um Sensor BNO055, que possui várias funções, dentre elas as de giroscópio e acelerômetro.

Sensores inerciais são dispositivos que desempenham um papel crucial na navegação autônoma de robôs. Eles se baseiam na primeira lei de Newton, a inércia, para medir as forças que atuam no robô, incluindo a aceleração, que é detectada pelos acelerômetros, e a velocidade angular, que é medida pelos giroscópios. Esses sensores são frequentemente integrados em uma única unidade conhecida como Unidade de Medição Inercial (IMU) \cite{Groves2013}.

Um dos sensores que formam uma IMU é o acelerômetro, de características proprioceptivo e passivo, seu funcionamento consiste em uma massa de prova de um material sensível à aceleração. Esse material é frequentemente um pequeno microchip chamado microeletromecânico (MEMS) em que há pequenas estruturas sensíveis à aceleração que se deformam quando o chip é acelerado alterando sua capacitância \cite{Lawrence1998}. A faixa de aceleração do acelerômetro é fornecida em função da gravidade, representada pela letra \textit{g}, onde \textit{1g} equivale à 9,80665 m/s. 

Outro componente crucial da IMU é o giroscópio, projetado para medir a taxa de rotação ou a velocidade angular do robô. Assim como o acelerômetro, o giroscópio faz parte da unidade de medição inercial e opera com base no princípio da conservação do momento angular. De acordo com esse princípio, um objeto em rotação tende a continuar girando a uma taxa constante, a menos que uma força externa atue sobre ele. O giroscópio contém um rotor (\ref{fig: giroscopio}) que gira a uma velocidade constante. Quando o robô gira ou vira, a tendência do rotor em manter sua orientação no espaço cria uma força que pode ser medida e usada para determinar a taxa de rotação \cite{Groves2013}.

\subsection{Raspberry Pi}

TODO: especificação rasp

\subsection{ESP32}

TODO: especificação esp

% capítulo 3 - metodologia
\chapter{Metodologia}

\section{Visão geral}

\textbf{Métodos:} A metodologia adotada para o desenvolvimento do robô equilibrista "Sir Galahad" segue as etapas abaixo:

\textbf{Definição de Requisitos:}
\begin{itemize}
   \item  Identificação dos requisitos funcionais e não-funcionais do robô, considerando busca, identificação e atingimento de alvos, além de equilíbrio dinâmico e precisão de movimento.
\end{itemize}

\textbf{Desenvolvimento do Software:}
\begin{itemize}
   \item Utilização da Raspberry Pi para implementação de partes do software de alto nível, utilizando a linguagem Python para o controle do robô e processamento de imagem.
   \item Integração do microcontrolador ESP32 para as camadas mais baixas do software, incluindo a leitura de dados dos sensores inerciais e controle dos motores.
\end{itemize}

\textbf{Modelagem Mecânica:}
\begin{itemize}
   \item Utilização do software SolidWorks para a modelagem 3D detalhada da estrutura do robô.
   \item Garantia de equilíbrio e resistência estrutural, além de validação da ergonomia e das dimensões do robô virtualmente antes da fabricação.
\end{itemize}

\textbf{Fabricação e Montagem:}
\begin{itemize}
   \item Fabricação da estrutura mecânica por meio de manufatura aditiva, assegurando leveza e precisão.
   \item Montagem dos componentes eletrônicos na estrutura física, levando em conta distribuição de peso e acessibilidade para manutenção.
\end{itemize}

\textbf{Projeto Eletrônico:}
\begin{itemize}
   \item Utilização do software Flux para projetar o circuito eletrônico, incluindo os componentes necessários para sensores, microcontroladores e motores.
   \item Garantia de conexões adequadas e otimização da eficiência energética.
\end{itemize}

\textbf{Testes e Ajustes:}
\begin{itemize}
   \item Realização de testes contínuos para verificar a capacidade de equilíbrio, precisão de movimento e detecção de alvos.
   \item Ajustes nos algoritmos de controle e nos parâmetros dos sensores para aprimorar o desempenho do robô.
\end{itemize}

\textbf{Validação e Avaliação:}
\begin{itemize}
   \item Validação final por meio de testes de campo e avaliação dos resultados obtidos em relação aos objetivos propostos.
   \item Comparação do desempenho do "Sir Galahad" com os critérios estabelecidos e considerações finais sobre o projeto.

\end{itemize} 

\section{Projeto mecânico}

TODO: inserir projeto mecânico

\section{Projeto de hardware}

TODO: inserir projeto de hardware e eletronica

* Obs.: nao esqueça de apresentar o diagrama de blocos do hardware. 


\section{Projeto de software}

 TODO: inserir aqui os diagramas de software

* Obs.: nao esqueça de apresentar os diagramas de estados (statecharts) do 
software.

\section{Integração}

TODO: discorrer sobre a integração

% capítulo 4 - experimentos
\chapter{Experimentos e resultados}

% testes, integração, etc.

TODO: testes e resultados obtidos

\section{Testes da Estrutura}

\section{Testes da Eletrônica}

\section{Testes do Software}

\subsection{Firmware}

\subsection{Visão Computacional}

\section{Testes de Integração}

% capítulo 5 - cronograma
\chapter{Cronograma e custos do projeto}

\section{Cronograma}

TODO: transcrever o cronograma para o relatório

As informações sobre o projeto estão centralizadas em sua página do \href{https://www.notion.so/Sir-Galahad-d9482e0c1ac040d4a7b6bf2bfb223bff}{Notion}, e o cronograma pode ser visto em detalhes por meio do \href{https://www.notion.so/Sir-Galahad-d9482e0c1ac040d4a7b6bf2bfb223bff}{\textcolor{blue}{link}}. 

\section{Custos}

Nesta seção, descrevemos os materiais e custos esperados, além da abordagem metodológica adotada para o desenvolvimento do robô equilibrista.

\textbf{Materiais:} Abaixo estão listados os materiais previstos para a realização do projeto, os itens destacados com (*) já são de posse dos membros do projeto, portanto não somaram aos custos esperados. 

\begin{tabframed}[h]%% Ambiente tabframed
   %\captionsetup{width=0.5\textwidth}%% Largura da legenda
   \caption{Materiais utilizados no desenvolvimento do sistema}%% Legenda
   \label{quad:exemplo1}%% Rótulo
   \renewcommand{\arraystretch}{1.5}
   \begin{tabular}{|l|l|l|l|l}
   \cline{1-4}
   \textbf{Materiais} & \textbf{Valor unitário (R\$)} & \textbf{Quantidade} & \textbf{Valor total (R\$)} \\ 
    
      \hline
      \href{https://produto.mercadolivre.com.br/MLB-2004934384-lip0-3s-111-v-1500-mah-3s-_JM?matt_tool=68334988&matt_word=&matt_source=google&matt_campaign_id=14300471977&matt_ad_group_id=127503848075&matt_match_type=&matt_network=g&matt_device=c&matt_creative=542969655996&matt_keyword=&matt_ad_position=&matt_ad_type=pla&matt_merchant_id=542516090&matt_product_id=MLB2004934384&matt_product_partition_id=1801247246545&matt_target_id=pla-1801247246545}{Power Banck 1000mAh} & 84,90 & 1 & 84,90 \\

      \hline 
      \href{https://www.aliexpress.com/item/32223093678.html?srcSns=sns_Copy&spreadType=socialShare&bizType=ProductDetail&social_params=21099436311&aff_fcid=c234ac5e17824f20a319f9cc17dbaf2f-1692194363304-03902-_mPjn7Ro&tt=MG&aff_fsk=_mPjn7Ro&aff_platform=default&sk=_mPjn7Ro&aff_trace_key=c234ac5e17824f20a319f9cc17dbaf2f-1692194363304-03902-_mPjn7Ro&shareId=21099436311&businessType=ProductDetail&platform=AE&terminal_id=676ed690bcd2403d8bbab55c9f2e36b3&afSmartRedirect=y}{Drive Motor DRV8833} & 2,61 & 2 &	5,22 \\
   
      \hline 
      \href{https://shopee.com.br/product/534679327/14372089266}{(*) Micro Servo MG90S}	& 13,00 & 1 & 13,00 \\

      \hline
      \href{https://www.makerhero.com/produto/raspberry-pi-4-model-b/}{(*) Raspberry Pi 4 Model B+} & 698,15 & 1 & 698,15 \\

      \hline
      \href{https://shopee.com.br/product/550918841/11054519654}{(*) Esp32} & 23,31 & 1 & 23,31 \\

      \hline
      Custos de impressão 3D & 75,00 & 1 & 75,00 \\

      \hline
      \href{https://pt.aliexpress.com/item/1005001279982165.html?srcSns=sns_Copy&spreadType=socialShare&bizType=ProductDetail&social_params=21106393547&aff_fcid=0a6046e63fd945d7a1d698fe1e129e32-1692194250357-01162-_mMVW6Jk&tt=MG&aff_fsk=_mMVW6Jk&aff_platform=default&sk=_mMVW6Jk&aff_trace_key=0a6046e63fd945d7a1d698fe1e129e32-1692194250357-01162-_mMVW6Jk&shareId=21106393547&businessType=ProductDetail&platform=AE&terminal_id=676ed690bcd2403d8bbab55c9f2e36b3&afSmartRedirect=y}{Motor com encoder e roda JGA25-370}	& 34,25	 & 2 & 68,50 \\

      \hline
      \href{https://produto.mercadolivre.com.br/MLB-2004934384-lip0-3s-111-v-1500-mah-3s-_JM?matt_tool=68334988&matt_word=&matt_source=google&matt_campaign_id=14300471977&matt_ad_group_id=127503848075&matt_match_type=&matt_network=g&matt_device=c&matt_creative=542969655996&matt_keyword=&matt_ad_position=&matt_ad_type=pla&matt_merchant_id=542516090&matt_product_id=MLB2004934384&matt_product_partition_id=1801247246545&matt_target_id=pla-1801247246545}{(*) Bateria LiPo 3s 1500 mAh} & 119,99 & 1 & 119,99 \\

      \hline
      \href{https://pt.aliexpress.com/item/1005005915264178.html?srcSns=sns_Copy&spreadType=socialShare&bizType=ProductDetail&social_params=21100114064&aff_fcid=9b62215bdd524891bba0b98b5bbb55e5-1692194179415-09393-_mKQfDDM&tt=MG&aff_fsk=_mKQfDDM&aff_platform=default&sk=_mKQfDDM&aff_trace_key=9b62215bdd524891bba0b98b5bbb55e5-1692194179415-09393-_mKQfDDM&shareId=21100114064&businessType=ProductDetail&platform=AE&terminal_id=676ed690bcd2403d8bbab55c9f2e36b3&afSmartRedirect=y}{Sensor Inercial BNO055 + BMP280, SEN0253} & 361,40 & 1 & 361,40 \\
    
      \hline
      \href{https://pt.aliexpress.com/item/1005003954117993.html?srcSns=sns_Copy&spreadType=socialShare&bizType=ProductDetail&social_params=21106405512&aff_fcid=ecdaa74a5af14122ba08116816bff22f-1692194368880-06711-_mttWAZ0&tt=MG&aff_fsk=_mttWAZ0&aff_platform=default&sk=_mttWAZ0&aff_trace_key=ecdaa74a5af14122ba08116816bff22f-1692194368880-06711-_mttWAZ0&shareId=21106405512&businessType=ProductDetail&platform=AE&terminal_id=676ed690bcd2403d8bbab55c9f2e36b3&afSmartRedirect=y}{Câmera}	& 59,32	& 1 & 59,32 \\

      \hline

   \end{tabular}
   \fonte{Autoria própria}%% Fonte
   \end{tabframed}


% capítulo 6 - conclusão 
\chapter{Conclusões}

\section{Conclusões}

TODO: conclusões

\section{Trabalhos futuros}

TODO: trabalhos futuros/melhorias para o robô

%ex.: melhorias,  entre outros.

\arquivosdereferencias{PosTextual/Referencias/library}

\end{document}